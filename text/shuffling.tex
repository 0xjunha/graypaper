\section{Shuffling}\label{sec:shuffle}

The Fisher-Yates shuffle function is defined formally as:
\begin{equation}\label{eq:suffle}
  \forall T, l \in \N: \mathcal{F}\colon\left\{\begin{aligned}
    (\seq{T}_l, \seq{\N}_{l:}) &\to \seq{T}_l\\
    (\mathbf{s}, \mathbf{r}) &\mapsto \begin{cases}
      [\mathbf{s}_{\mathbf{r}_0 \bmod l}] \frown \mathcal{F}([\mathbf{s}_i\mid i \orderedin \N_l \setminus \{\mathbf{r}_0 \bmod l\}], [\mathbf{r}_1, \mathbf{r}_2, \dots]) &\when \mathbf{s} \ne []\\
      [] &\otherwise
    \end{cases}
  \end{aligned}\right.
\end{equation}

Since it is often useful to shuffle a sequence based on some random seed in the form of a hash, we provide a secondary form of the shuffle function $\mathcal{F}$ which accepts a 32-byte hash instead of the numeric sequence. We define $\mathcal{Q}$, the numeric-sequence-from-hash function, thus:
\begin{align}
  \forall l \in \N:\ \mathcal{Q}_l&\colon\left\{\begin{aligned}
    \H &\to \seq{\N_l}\\
    h &\mapsto [\de_4(\mathcal{H}(h \frown \se_4(\lfloor \nicefrac{i}{8}\rfloor))_{4i \bmod 32 \dots}) \mid i \in \N_l]
  \end{aligned}\right.\\
  \label{eq:sequencefromhash}
  \forall T, l \in \N:\ \mathcal{F}&\colon\left\{\begin{aligned}
    (\seq{T}_l, \H) &\to \seq{T}_l\\
    (\mathbf{s}, h) &\mapsto \mathcal{F}(\mathbf{s}, \mathcal{Q}_l(h))
  \end{aligned}\right.
\end{align}

\section{General Merklization}\label{sec:merklization}

\subsection{Binary Merkle Trees}

The Merkle tree is a cryptographic data structure yielding a hash commitment to a specific sequence of values. It provides $O(N)$ computation and $O(\log(N))$ proof size for inclusion. This \emph{well-balanced} formulation ensures that the maximum depth of any leaf is minimal and that the number of leaves at that depth is also minimal.

\subsubsection{Well-balanced tree}
We define the well-balanced binary Merkle function as $\mathcal{M}_2$:
\begin{equation}
  \begin{aligned}
    \mathcal{M}_2&\colon \left\{\begin{aligned}
      (\seq{\Y_{n \ne 32}}, \Y \to \H) &\to \H \\
      (\mathbf{v}, H) &\mapsto \begin{cases}
        \H_0 &\when |\mathbf{v}| = 0\\
        H(\mathbf{v}_0) &\when |\mathbf{v}| = 1 \\
        N(\mathbf{v}, H) &\otherwise
      \end{cases} \\
    \end{aligned}\right.\label{eq:binarymerkleroot} \\
    \where N&\colon\left\{\begin{aligned}
      (\seq{\Y_{n \ne 32}}_{1:}, \Y \to \H) &\to \Y \\
      (\mathbf{v}, H) &\mapsto \begin{cases}
        \mathbf{v}_0 &\when |\mathbf{v}| = 1 \\
        H(N(\mathbf{v}_{\dots\ceil{\nicefrac{|\mathbf{v}|}{2}}}) \concat N(\mathbf{v}_{\ceil{\nicefrac{|\mathbf{v}|}{2}}\dots})) &\otherwise
      \end{cases}
    \end{aligned}\right.
  \end{aligned}
\end{equation}

\subsubsection{Constant-depth tree}
We define the constant-depth binary Merkle function as $\mathcal{M}_2^*$:
\begin{equation}
  \begin{aligned}
    \mathcal{M}_2^*&\colon \left\{\begin{aligned}
      (\seq{\Y_{n \ne 32}}, \Y \to \H) &\to \H\\
      (\mathbf{v}, H) &\mapsto \mathcal{M}_2(\mathbf{v}', H)\ \where |\mathbf{v}'| = 2^{\ceil{\log_2(|\mathbf{v}|)}}\,,\ \mathbf{v}'_i = \begin{cases}
        \mathbf{v}_i &\when i < |\mathbf{v}|\\
        [] &\otherwise
      \end{cases}\\
    \end{aligned}\right.
  \end{aligned}
\end{equation}

\subsection{Merkle Mountain Ranges}\label{sec:mmr}

The Merkle mountain range (\textsc{mmr}) is an append-only cryptographic data structure which yields a commitment to a sequence of values. Appending to an \textsc{mmr} and proof of inclusion of some item within it are both $O(log N)$ in time and space for the size of the set.

We define a Merkle mountain range as being within the set $\seq{\H?}$, a sequence of peaks, each peak the root of a Merkle tree containing $2^i$ items where $i$ is the index in the sequence. Since we support set sizes which are not always powers-of-two-minus-one, some peaks may be empty, $\none$ rather than a Merkle root.

Since the sequence of hashes is somewhat unwieldy as a commitment, Merkle mountain ranges are themselves generally hashed before being published. Hashing them removes the possibility of further appending so the range itself is kept on the system which needs to generate future proofs.

\newcommand*{\deffunc}[1]{\left\{\,\begin{aligned}#1\vphantom{x'_i}\end{aligned}\right.}

We define the append function $\mathcal{A}$ as:
\begin{equation}
  \begin{aligned}
    \label{eq:mmrappend}
    \mathcal{A}&\colon\deffunc{
      (\seq{\H?}, \H, \Y\to\H) &\to \seq{\H?}\\
      (\mathbf{r}, l, H) &\mapsto P(\mathbf{r}, l, 0, H)
    }\\
    \where P&\colon\deffunc{
      (\seq{\H?}, \H, \N, \Y\to\H) &\to \seq{\H?}\\
      (\mathbf{r}, l, n, H) &\mapsto \begin{cases}
        \mathbf{r} \doubleplus l &\when n \ge |\mathbf{r}|\\
        R(\mathbf{r}, n, l) &\when n < |\mathbf{r}| \wedge \mathbf{r}_n = \none\\
        P(R(\mathbf{r}, n, \none), H(\mathbf{r}_n \concat l), n + 1, H) &\otherwise
      \end{cases}
    }\\
    \also R&\colon\deffunc{
      (\seq{T}, \N, T) &\to \seq{T}\\
      (\mathbf{s}, i, v) &\mapsto \mathbf{s}'\ \where \mathbf{s}' = \mathbf{s} \exc \mathbf{s}'_i = v
    }
  \end{aligned}
\end{equation}

We define the \textsc{mmr} encoding function as $\se_M$:
\begin{equation}
  \se_M\colon\deffunc{
    \seq{\H?} &\to \H \\
    \mathbf{b} &\mapsto \se(\var{[\maybe{x}\mid x \orderedin \mathbf{b}]})
  }
\end{equation}

%\begin{equation}
%  \left\{\mathbb{H}\right.
%\end{equation}
%\begin{equation}
%  \left\{\,\begin{aligned} & (\mathbb{H}) \\ & \begin{cases}x&y\end{cases} \end{aligned}\right.
%\end{equation}
