\section{Shuffling}\label{sec:shuffle}

The Fisher-Yates shuffle function is defined formally as:
\begin{equation}\label{eq:suffle}
  \forall T, l \in \N: \mathcal{F}\colon\left\{\begin{aligned}
    (\seq{T}_l, \seq{\N}_{l:}) &\to \seq{T}_l\\
    (\mathbf{s}, \mathbf{r}) &\mapsto \begin{cases}
      [\mathbf{s}_{\mathbf{r}_0 \bmod l}] \frown \mathcal{F}(\mathbf{s}'_{\dots l-1}, \mathbf{r}_{1\dots})\ \where \mathbf{s}' = \mathbf{s} \exc \mathbf{s'}_{\mathbf{r}_0 \bmod l} = \mathbf{s}_{l - 1} &\when \mathbf{s} \ne []\\
      [] &\otherwise
    \end{cases}
  \end{aligned}\right.
\end{equation}

Since it is often useful to shuffle a sequence based on some random seed in the form of a hash, we provide a secondary form of the shuffle function $\mathcal{F}$ which accepts a 32-byte hash instead of the numeric sequence. We define $\mathcal{Q}$, the numeric-sequence-from-hash function, thus:
\begin{align}
  \forall l \in \N:\ \mathcal{Q}_l&\colon\left\{\begin{aligned}
    \H &\to \seq{\N_{2^{32}}}_l\\
    h &\mapsto [\de_4(\hash(h \frown \se_4(\lfloor \nicefrac{i}{8}\rfloor))_{4i \bmod 32 \dots+4}) \mid i \orderedin \N_l]
  \end{aligned}\right.\\
  \label{eq:sequencefromhash}
  \forall T, l \in \N:\ \mathcal{F}&\colon\left\{\begin{aligned}
    (\seq{T}_l, \H) &\to \seq{T}_l\\
    (\mathbf{s}, h) &\mapsto \mathcal{F}(\mathbf{s}, \mathcal{Q}_l(h))
  \end{aligned}\right.
\end{align}
