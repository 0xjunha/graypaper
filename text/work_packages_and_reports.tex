\section{Work Packages and Work Reports}\label{sec:workpackagesandworkreports}

\subsection{Honest Behavior}

We have so far specified how to recognize blocks for a correctly transitioning \Jam blockchain. Through defining the state transition function and a state Merklization function, we have also defined how to recognize a valid header. While it is not especially difficult to understand how a new block may be authored for any node which controls a key which would allow the creation of the two signatures in the header, nor indeed to fill in the other header fields, readers will note that the contents of the extrinsic remain unclear.

We define not only correct behavior through the creation of correct blocks but also \emph{honest behavior}, which involves the node taking part in several \emph{off-chain} activities. This does have analogous aspects within \emph{YP} Ethereum, though it is not mentioned so explicitly in said document: the creation of blocks along with the gossiping and inclusion of transactions within those blocks would all count as off-chain activities for which honest behavior is helpful. In \Jam's case, honest behavior is well-defined and expected of at least $\nicefrac{2}{3}$ of validators.

Beyond the production of blocks, incentivized honest behavior includes:
\begin{itemize}
    \item the guaranteeing and reporting of work-packages, along with chunking and distribution of both the chunks and the work-package itself, discussed in section \ref{sec:guaranteeing};
    \item assuring the availability of work-packages after being in receipt of their data;
    \item determining which work-reports to audit, fetching and auditing them, and creating and distributing judgements appropriately based on the outcome of the audit;
    \item submitting the correct amount of work seen being done by other validators, discussed in section \ref{sec:ratings}.
\end{itemize}

\subsection{Segments and the Manifest}

Our basic erasure-coding segment size is $\mathsf{W}_C = 2^{12}$, derived from practical concerns over the need to support 1,023 participants and a desire to ensure the possibility of high performance using the \textsc{fft} algorithm and \textsc{simd} \textsc{cpu} instruction. A data segment, defined as the set $\G$, is an octet sequence of fixed length $\mathsf{W}_S = 2^{12}$. It is the smallest datum which may individually be introduced by, or transmitted between, work packages. Being equivalent to the erasure-coding segment size ensures that the data segments of work package can be efficiently placed in the distributed availability system.
\begin{equation}\label{eq:segment}
  \G \equiv \Y_{\mathsf{W}_S}
\end{equation}

Work-packages only ever \emph{reference} segments through commitments; they do not include the data directly. By splitting data up into segments, data introduced to---or generated by---a work-package may be utilized by other, later, work-packages without the need to re-introduce it manually. This is a key feature of the \Jam availability system.

Work-packages have three types of data associated with them (other than its own inherent data): A cryptographic commitment to each \emph{imported} segment, another commitment to each \emph{extrinsic} segment and finally the number of segments which are \emph{exported}.

Extrinsic segments are segments which are being introduced into the system by the work-package itself. They are exposed to the Refine logic as an argument and may be drawn upon by later work packages if they are referenced in imports. We commit to them through including each of their hashes in the work-package.

Exported segments, meanwhile, are segments which are \emph{generated} through the execution of the Refine logic and thus are a side effect of transforming the work-package into a work-report. Since their data is deterministic based on the execution of the Refine logic, we do not require any particular commitment to them in the work-package beyond knowing how many are associated with each Refine invocation in order that we can supply an exact index.

Finally, imported segments are segments which were introduced by previous work-packages as either extrinsic or exported segments. In order to them to be easily fetched and verified they are referenced not by hash but rather the root of a Merkle tree which includes any other segments introduced in the same way at the time, together with an index into this sequence. This allows for justifications of correctness to be generated, stored, included alongside the fetched data and verified. This is described in depth in the next section.

\subsubsection{Data Collection and Justification}

It is the task of a guarantor to reconstitute all imported segments through fetching said segments' erasure-coded chunks from enough unique validators. Reconstitution alone is not enough since corruption of the data would occur if one or more validators provided an incorrect chunk. For this reason we ensure that the import segment specification (a Merkle root and an index into the tree) be a kind of cryptographic commitment capable of having a justification applied to demonstrate that some particular segment is indeed correct.

Justification data must be available to any node over the course of its segment's potential requirement. At around 350 bytes to justify a single segment, justification data is too voluminous to have all validators store all data. We therefore use the same overall availability framework for hosting justification metadata as the data itself.

For the guarantor to justify to themselves is important as it ensures that they are not wasting their time on potentially Byzantine behavior. However, quantitatively more important is that auditors be able to justify as cheaply as possible, since guaranteeing happens twice or thrice but auditing happens on average 30 times.

The same is true for imported segments. Thus, for both import segments and justification data, guarantors take time to reconstruct and justify said data (though with a good network protocol, only a single guarantor should generally need to do this for any work-package) before placing it back into the availability system in a most convenient format, whereas auditors are spoon-fed the import data as well as justification of its correctness. Data is thus duplicated over two availability contexts: one long-lived and more compressed representation meant for guarantors, and a second, short-lived, more readily usable representation meant for auditors.

\subsection{Packages and Items}\label{sec:packagesanditems}

% TODO: Complete rename to \mathbf{c}
\newcommand{\Tcontext}{\mathbf{x}}

We begin by defining a \emph{work-package}, of set $\mathbb{P}$, and its constituent \emph{work item}s, of set $\mathbb{I}$. A work-package includes a simple blob acting as an authorization token $\mathbf{j}$, the index of the service which hosts the authorization code $h$, an authorization code hash $c$ and a parameterization blob $\mathbf{p}$, a context $\Tcontext$ and a sequence of work items $\mathbf{i}$:
\begin{equation}\label{eq:workpackage}
  \mathbb{P} \equiv \tuple{\;\begin{aligned}
    &\isa{\mathbf{j}}{\Y},\ \isa{h}{\N_\mathsf{S}},\ \isa{c}{\H},\\
    &\isa{\mathbf{p}}{\Y},\ \isa{\Tcontext}{\mathbb{X}},\ \isa{\mathbf{i}}{\seq{\mathbb{I}}_{1:\mathsf{I}}}
  \end{aligned}}
\end{equation}

A work item includes: $s$ the identifier of the service to which it relates, the code hash of the service at the time of reporting $c$ (whose preimage must be available from the perspective of the lookup anchor block), a payload blob $y$, a gas limit $g$, and the three elements of its manifest, a sequence of imported data segments $\mathbf{i}$ identified by the root of the \emph{segments tree} and an index into it, $\mathbf{x}$, a sequence of hashed of data segments to be introduced in this block (and which we assume the validator knows) and $e$ the number of data segments exported by this work item:
\begin{equation}\label{eq:workitem}
    \mathbb{I} \equiv \tuple{\begin{aligned}
      &\isa{s}{\N_\mathsf{S}} \ts
      \isa{c}{\H} \ts
      \isa{\mathbf{y}}{\Y} \ts
      \isa{g}{\N_G} \ts \\
      &\isa{\mathbf{i}}{\seq{\tuple{\H,\N}}} \ts
      \isa{\mathbf{x}}{\seq{\H}} \ts
      \isa{e}{\N}
    \end{aligned}}
\end{equation}

We limit the total number of exported and extrinsic items to $\mathsf{W}_M = 2^{11}$. We also place the same limit on the total number of extrinsic and imported items:
\begin{align}
  \forall p \in \mathbb{P}: \sum_{i \in p_\mathbf{i}} (i_e + |i_\mathbf{x}|) &\le \mathsf{W}_M \  \wedge \  \sum_{i \in p_\mathbf{i}} (|i_\mathbf{i}| + |i_\mathbf{x}|) \le \mathsf{W}_M
\end{align}

We make an assumption that the preimage to each extrinsic hash in each work-item is known by the guarantor. In general this data will be passed to the guarantor alongside the work-package.
%
%We limit the encoded size of a work-package plus the total size of the implied import and extrinsic items to 15\textsc{mb} in order to allow for around 2.5\textsc{mb}/s/core data throughput:
%\begin{align}
%  \label{eq:checkextractsize}\forall p &\in \mathbf{P}: \left(|\se(p)| + \sum_{i \in p_\mathbf{o}}(i_l) + \sum_{\mathcal{H}(\mathbf{x}) \in p_\mathbf{n}}(|\mathbf{x}|) \right) \le \mathsf{W}_P\\
%  \mathsf{W}_P &= 15\cdot2^{20}
%\end{align}
%
%The implication of equation \ref{eq:checkextractsize} is that we have access to the preimage of all extrinsic data. For guaranteeing, this implies the work-package author probably submits the preimages alongside the work-package itself. For auditing, the extrinsic preimages may be reconstructed in the same manner as the work-package. Both are described later.

We define the item-to-result function $\Gamma$ as:
\begin{equation}
  \Gamma\colon\left\{\begin{aligned}
    (\mathbb{I}, \Y \cup \mathbb{J}) &\to \mathbb{L}\\
    ((s, c, \mathbf{y}, g), o) &\mapsto (s, c, \mathcal{H}(\mathbf{y}), g, o)
  \end{aligned}\right.
\end{equation}

We define the work-package's implied authorizer as $\mathbf{p}_a$, the hash of the concatenation of the authorization code and the parameterization. We define the authorization code as $\mathbf{p}_\mathbf{c}$ and require that it be available at the time of the lookup anchor block from the historical lookup of service $h$. Formally:
\begin{equation}
  \forall \mathbf{p} \in \mathbb{P}: \left\{\,\begin{aligned}
    \mathbf{p}_a &\equiv \mathcal{H}(\mathbf{p}_\mathbf{c} \concat \mathbf{p}_\mathbf{p}) \\
    \mathbf{p}_\mathbf{c} &\equiv \Lambda(\delta[\mathbf{p}_h], (\mathbf{p}_x)_t, \mathbf{p}_c) \\
    \mathbf{p}_\mathbf{c} &\in \Y
  \end{aligned}\right.
\end{equation}

($\Lambda$ is the historical lookup function defined in equation \ref{eq:historicallookup}.)

\subsubsection{Yielding}
A work-package \emph{yields} segments which are exported from, or extrinsic to, any of its work-items. Two separate \emph{segments-roots} are placed in the work-report, one for extrinsic segments yielded by work-items, and the other for exported segments yielded. Both are ordered according to the work-item which is yielding. They're both formed as the root of a constant-depth binary Merkle tree as defined in equation \ref{eq:constantdepthmerkleroot}:

In addition to the segments-roots, guarantors are required to erasure-code and distribute four data sets: the encoded work-package itself, together with self-justifying imported segments, extrinsic segments and exported segments, the latter two including \emph{paged-proofs}. The first two items are short-lived; assurers are expected to keep them only until they believe there is general confidence in the validity of the work-result. Erasure-coded chunks of the latter two are long-lived and expected to be kept for a minimum of 28 days (672 complete epochs) following the reporting of the work-report.

Extrinsic data is reconstructed by both auditors and any later guarantors who are called upon to import this data for use in a work-package. Imported segment data may be reconstructed only by auditors, and exported segment data may be used only for later guarantors since auditors can derive exported segments perfectly through Refine execution alone.

We define the paged-proofs function $P$ which accepts a series of segments $\mathbf{s}$ and defines some series of additional segments placed into the availability system for erasure-coding and distribution. The function evaluates to pages of hashes, together with subtree proofs, such that justifications of correctness based on a yield root may be made from it:
\begin{equation}\label{eq:pagedproofs}
  Y\colon\left\{\begin{aligned}
    \seq{\G} &\to \seq{\G} \\
    \mathbf{s} &\mapsto [\mathcal{P}_{\mathsf{W}_S}(\mathcal{J}_6(\mathbf{s}, i) \concat \wideparen{\mathbf{s}_{i\dots+64}}) \mid i \orderedin 64\cdot\N_{\ceil{\nicefrac{|\mathbf{s}|}{64}}}]
  \end{aligned}\right.
\end{equation}

Note: in the case that $|\mathbf{s}|$ is not a multiple of 64, then the term $\mathbf{s}_{i\dots+64}$ will correctly refer to fewer than 64 elements if it is the final page.

\subsection{Computation of Work Results}\label{sec:computeworkresult}

We now come to the work result computation function $\Xi$. This forms the basis for all utilization of cores on \Jam. It accepts some work-package $\mathbf{p}$ for some nominated core $c$ and results in either an error $\error$ or the work result and series of exported segments. This function is deterministic and requires only that it be evaluated within eight epochs of a recently finalized block thanks to the historical lookup functionality. It can thus comfortably be evaluated by any node within the auditing period, even allowing for practicalities of imperfect synchronization.

Formally:
\begin{equation}\label{eq:workresultfunction}
  \!\!\Xi \colon \left\{\begin{aligned}
    (\mathbb{P}, \N_C) &\to \mathbb{W} \\
    (\mathbf{p}, c) &\mapsto \begin{cases}
        \error &\when \mathbf{o} \not\in \Y \\
        \tup{\is{a}{\mathbf{p}_a}, \mathbf{o}, \is{x}{\mathbf{p}_x}, s, r, q, v} \!\!\!\!\!&\otherwise
    \end{cases}
  \end{aligned}\right.\!\!\!\!\!\!\!
\end{equation}

where:
\begin{align*}
  \mathbf{o} &= \Psi_I(\mathbf{p}, c) \\
  (r, \overline{\mathbf{e}}) &= \transpose[(\Gamma(\mathbf{p}_\mathbf{i}[j], r), \mathbf{e}) \mid (r, \mathbf{e}) = I(\mathbf{p}, j), j \orderedin \N_{|\mathbf{p}_\mathbf{i}|}] \\
  % r here should be bold.
  I(\mathbf{p}, j) &\equiv R(\mathbf{p}, \mathbf{p}_\mathbf{i}[j], \sum_{k < j}\mathbf{p}_\mathbf{i}[k]_e)\\
  R(\mathbf{p}, i, \ell) &\equiv \Psi_R(i_c, i_g, i_s, \mathcal{H}(\mathbf{p}), i_\mathbf{y}, \mathbf{p}_\Tcontext, \mathbf{p}_a, \mathbf{i}, \mathbf{x}, \ell)\\
  \mathbf{i} &= [\overline{\mathbf{s}}[j] \mid (\mathcal{M}(\overline{\mathbf{s}}), j) \orderedin i_\mathbf{i}, i \orderedin \mathbf{p}_\mathbf{i}]\\
  \mathbf{x} &= [\mathbf{s} \mid \mathcal{H}(\mathbf{s}) \orderedin i_\mathbf{x}, i \orderedin \mathbf{p}_\mathbf{i}]\\
  s &= A(\se(\mathbf{p}), \wideparen{\mathbf{i}} \frown \wideparen{\mathbf{j}}, \mathbf{x}, \wideparen{\overline{\mathbf{e}}}) \\
  \mathbf{j} &= [\mathcal{J}(\mathbf{s}, n) \mid (\mathcal{M}(\mathbf{s}), n) \orderedin i_\mathbf{i}, i \orderedin \mathbf{p}_\mathbf{i}]
\end{align*}

Note that while $\mathbf{i}$ and $\mathbf{j}$ are both formulated using the term $\mathbf{s}$ (the full set of segments yielded alongside each segment to be imported) this value is not generally needed as the justification can be derived through a single paged-proof. This reduces the worst case data fetching for a guarantor to two segments for every one to be imported. In the case that contiguously exported segments are imported (which we might assume is a fairly common situation), then a single proof-page should be sufficient to justify many imported segments.

The Is Authorized logic it references is first executed to ensure that the work-package warrants the needed core-time. Next, the guarantor should ensure that all segments-tree roots which form imported segment commitments are known and have not expired. Finally, the guarantor should ensure that they can fetch all preimage data referenced as the commitments of extrinsic segments.

Once done, then imported segments must be reconstructed; this implies, for each imported segment, fetching erasure-coded chunks from enough unique validators (342, specifically) and is described in more depth in appendix \ref{sec:erasurecoding}. (Since we specify systematic erasure-coding, its reconstruction is trivial in the case that the correct 342 validators are responsive.) Chunks must be fetched for both the data itself and for justification metadata which allows us to ensure that the data is correct.

Validators, in their role as availability assurers, should index such chunks according to the index of the segments-tree whose reconstruction they facilitate. Since the data for segment chunks is so small at 12 bytes, fixed communications costs should be kept to a bare minimum. A good network protocol (out of scope at present) will allow guarantors to specify only the segments-tree root and index together with a Boolean to indicate whether the proof chunk need be supplied. Since we assume at least 342 validators are online and benevolent, we can assume that the guarantor can compute $\mathbf{i}$ and $\mathbf{j}$ above with confidence, based on the general availability of data committed to with $\mathbf{x}^\clubsuit$ and $\mathbf{e}^\clubsuit$, which are specified below.

%TODO: Make result an error if the number of exported segments > stated number of exports. (If less, then additional ones are assumed to be zeroed.)

We define the segment-root function used for making a commitment to each of the two sequences for extrinsic segments and exported segments. The sequences are simply the sequences from each work-item concatenated together in the order of said work-items.

We define the availability specifier function $A$, which creates an availability specifier from an octet sequence of the work-package, an octet sequence of the concatenated import segments along with their proofs of correctness, and two sequences of segments for each extrinsic and exported segment alluded to in the manifest:
\begin{equation}
  \!\!\!A\colon\left\{\,\begin{aligned}
    &\tuple{\Y, \seq{\G}, \seq{\G}, \seq{\G}} \to \mathbb{S}\\
    &\tup{\mathbf{p}, \mathbf{i}, \mathbf{x}, \mathbf{e}} \mapsto \tup{h,\is{l}{|\mathbf{p}|},u,\is{x}{\mathcal{M}(\mathbf{x})},\ \is{e}{\mathcal{M}(\mathbf{e})}
    }
  \end{aligned}\right.\!\!\!\!\!
\end{equation}
\begin{align*}
  \where u &= \mathcal{M}_S(
    [\wideparen{\mathbf{j}} \mid \mathbf{j} \in \transpose[\mathbf{p}^\clubsuit, \mathbf{i}^\clubsuit, \mathbf{x}^\clubsuit, \mathbf{e}^\clubsuit]]
  )\\
  \also \mathbf{p}^\clubsuit &= \mathcal{H}^\#(\mathcal{C}(\mathcal{P}_{\mathsf{W}_C}(\mathbf{p})))\\
  \also \mathbf{i}^\clubsuit &= \mathcal{H}^\#(\mathcal{C}(\mathcal{P}_{\mathsf{W}_C}(\mathbf{i})))\\
  \also \mathbf{x}^\clubsuit &= \mathcal{M}_S^\#(\transpose\mathcal{C}^\#(\mathbf{x} \concat P(\mathbf{x}))) \\
  \also \mathbf{e}^\clubsuit &= \mathcal{M}_S^\#(\transpose\mathcal{C}^\#(\mathbf{e} \concat P(\mathbf{e})))
\end{align*}

The paged-proofs function $P$, defined earlier in equation \ref{eq:pagedproofs}, accepts a sequence of segments and returns a sequence of paged-proofs sufficient to justify the correctness of every segment. There are exactly $\ceil{\nicefrac{1}{64}}$ paged-proof segments as the number of yielded segments, each composed of a page of 64 hashes of segments, together with a Merkle proof from the root to the subtree-root which includes those 64 segments.

The functions $\mathcal{M}$ and $\mathcal{M}_S$ are the fixed-depth and simple binary Merkle root functions, defined in equations \ref{eq:constantdepthmerkleroot} and \ref{eq:simplemerkleroot}. The function $\mathcal{C}$ is the erasure-coding function, defined in appendix \ref{sec:erasurecoding}.

And $\mathcal{P}$ is the zero-padding function to take an octet array to some multiple of $n$ in length:
\begin{equation}\label{eq:zeropadding}
  \mathcal{P}_{n \in \N_{1:}}\colon\left\{\begin{aligned}
    \Y &\to \Y_{k\cdot n}\\
    \mathbf{x} &\mapsto \mathbf{x} \concat [0, 0, ...]_{((|x|+n - 1) \bmod n) + 1 \dots n}
  \end{aligned}\right.
\end{equation}

Validators are incentivized to distribute each newly erasure-coded data chunk to the relevant validator, since they are not paid for guaranteeing unless a work-report is considered to be \emph{available} by a super-majority of validators. Given our work-package $\mathbf{p}$, we should therefore send work-package chunk $\mathcal{C}(\mathcal{P}(\se(\mathbf{p}))_{\mathsf{W}_C})_v$ to each validator whose keys are $\kappa_v$ together with similarly corresponding chunks for imported, extrinsic and exported segments data, such that each validator can justify completeness according to the work-report's \emph{erasure-root}. In the case of a coming epoch change, they may also maximize expected reward by distributing to the new validator set (and thus also send the chunk to $(\gamma_\mathbf{k})_v$).

We will see this function utilized in the next sections, for guaranteeing, auditing and judging.
