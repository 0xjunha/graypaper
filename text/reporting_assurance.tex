\section{Reporting and Assurance}\label{sec:reporting}

Reporting and assurance are the two on-chain processes we do to allow the results of in-core computation to make its way into the service state singleton, $\delta$. A \emph{work-package}, which comprises several \emph{work items}, is transformed by validators acting as \emph{guarantors} into its corresponding \emph{work-report}, which similarly comprises several \emph{work outputs} and then presented on-chain within the \emph{guarantees} extrinsic. At this point, the work-package is erasure coded into a multitude of segments and each segment distributed to the associated validator who then attests to its availability through an \emph{assurance} placed on-chain. After either enough assurances or a time-out (whichever happens first), the work-report is considered \emph{available}, and the work outputs transform the state of their associated service by virtue of accumulation, covered in section \ref{sec:accumulation}.

From the perspective of the work-report, therefore, the guarantee happens first and the assurance afterwards. However, from the perspective of a block's state-transition, the assurances are best processed first since each core may only have a single work-report pending its package becoming available at a time. Thus, we will first cover the transition arising from processing the availability assurances followed by the work-report guarantees. This synchroneity can be seen formally through the requirement of an intermediate state $\rho^\ddagger$, utilized later in equation \ref{eq:reportcoresareunusedortimedout}.









\subsection{State}
The state of the reporting and availability portion of the protocol is largely contained within $\rho$, which tracks the work-reports which have been reported but not yet accumulated and the identities of the guarantors who reported them and the time at which it was reported. As mentioned earlier, at only one report may be assigned to a core at any given time. Formally:
\begin{equation}\label{eq:reportingstate}
  \rho \in \seq{\tuple{\isa{w}{\mathbb{W}},\isa{g}{\seq{\H_E}_{2:3}},\isa{t}{\N_T}}\bm{?}}_\mathsf{C}
\end{equation}

As usual, intermediate and posterior values ($\rho^\dagger$, $\rho^\ddagger$, $\rho'$) are held under the same constraints as the prior value.

\subsubsection{Work Report}\label{sec:workreport}
A work-report, of the set $\mathbb{W}$, is defined as a tuple of authorizer hash and output, the refinement context, the package specification, the availability specifier manifest roots for requirements ($q$) and provisions ($v$), and the results of the evaluation of each of the items in the package, which is always at least one item and may be no more than $\mathsf{I}$ items. Formally:
\begin{equation}\label{eq:workreport}
  \mathbb{W} \equiv \tuple{
    \begin{aligned}
      &\isa{a}{\H}\ts
      \isa{o}{\Y}\ts
      \isa{x}{\mathbb{X}}\ts
      % TODO: ^^^ Rename to c
      \isa{s}{\mathbb{S}}\ts\\
      &\isa{t}{\H}\ts
      % TODO: ^^^ Rename to x
      \isa{e}{\H}\ts
      \isa{r}{\seq{\mathbb{L}}_{1:\mathsf{I}}}
    \end{aligned}
  }\!\!\!\!\!
  % TODO: This should be bold o!!
  % TODO: This should be bold r!!
\end{equation}

The total serialized size of a work-report may be no greater than $\mathsf{W}_R$ bytes:
\begin{equation}
  \forall w \in \mathbb{W} : |\se(w)| \leq \mathsf{W}_R
\end{equation}


\subsubsection{Refinement Context}
A \emph{refinement context}, denoted by the set $\mathbb{X}$, describes the context of the chain at the point that the report's corresponding work-package was evaluated. It identifies two historical blocks, the \emph{anchor}, header hash $a$ along with its associated posterior state-root $s$ and posterior \textsc{Beefy} root $b$; and the \emph{lookup-anchor}, header hash $l$ and of timeslot $t$. Finally, it identifies the hash of an optional prerequisite work-package $p$. Formally:
\begin{equation}
  \mathbb{X} \equiv \ltuple\,\begin{alignedat}{7}
    \isa{a&}{\H}\ts\quad \isa{&s&}{\H}\ts\quad \isa{&b}{\H}\ts\\
    \isa{l&}{\H}\ts\quad \isa{&t&}{\N_T}\ts\quad \isa{&p}{\H\bm{?}}
  \end{alignedat}\rtuple
\end{equation}

\subsubsection{Availability}
We define the set of \emph{availability specifications}, $\mathbb{S}$, as the tuple of the work-package's hash and length together with an erasure-root and two segment-roots, one for the extrinsic segments, $x$, and one for the export segments, $e$. Work-results include the availability specification for the encoding of the corresponding work-package. Formally:
\begin{align}
  \mathbb{S} &\equiv \tuple{
    \isa{h}{\H}\ts
    \isa{l}{\N_L}\ts
    \isa{u}{\H}\ts
    \isa{x}{\H}\ts
    \isa{e}{\H}
  }
\end{align}

The \emph{availability manifest root} ($u$) is the root of a binary Merkle tree populated with the encoded availability specifications of each of the items in the work-package manifest, together with additional items exported from the work-package's refinement. It is used to verify the availability of all data required to audit the work-result and is discussed properly in section \ref{sec:workpackagesandworkreports}.

The two \emph{segment roots} ($x$ and $e$) are the roots of constant-depth, left-biased and zero-hash-padded binary Merkle trees populated with the encoded segments of each work-item, one for the extrinsic segments and one for the export segments. These are used to verify the correctness of the segments should they later be imported by some other work package and are discussed in section \ref{sec:workpackagesandworkreports}.

\subsubsection{Work Result}
We finally come to define a \emph{work result}, $\mathbb{L}$, which is the data conduit by which services' states may be altered through the computation done within a work-package.
\begin{align}\label{eq:workresult}
  \mathbb{L} &\equiv (s \in \N_S, c \in \H, l \in \H, g \in \Z_G, o \in \Y \cup \mathbb{J})
\end{align}

Work results are a tuple comprising several items. Firstly $s$, the index of the service whose state is to be altered and thus whose refine code was already executed. We include the hash of the code of the service at the time of being reported $c$, which must be accurately predicted within the work-report according to equation \ref{eq:reportcodesarecorrect};

Next, the hash of the payload ($l$) within the work item which was executed in the refine stage to give this result. This has no immediate relevance, but is something provided to the accumulation logic of the service. We follow with the gas prioritization ratio $g$ used when determining how much gas should be allocated to execute of this item's accumulate.

Finally, there is the output or error of the execution of the code $o$, which may be either an octet sequence in case it was successful, or a member of the set $\mathbb{J}$, if not. This latter set is defined as the set of possible errors, formally:
\begin{align}
  \mathbb{J} \in \{ \oog, \panic, \token{BAD}, \token{BIG} \}
\end{align}

The first two are special values concerning execution of the virtual machine, $\oog$ denoting an out-of-gas error and $\panic$ denoting an unexpected program termination. Of the remaining two, the first indicates that the service's code was not available for lookup in state at the posterior state of the lookup-anchor block. The second indicates that the code was available but was beyond the maximum size allowed $\mathsf{S}$.









\subsection{Package Availability Assurances}

We first define $\rho^\ddagger$, the intermediate state to be utilized next in section \ref{sec:workreportguarantees} as well as $\mathbf{R}$, the set of available work-reports, which will we utilize later in section \ref{sec:accumulation}. Both require the integration of information from the assurances extrinsic $\mathbf{E}_A$.

\subsubsection{The Assurances Extrinsic}
The assurances extrinsic is a sequence of \emph{assurance} values, at most one per validator. Each assurance is a sequence of binary values (\ie a bitstring), one per core, together with a signature and the index of the validator who is assuring. A value of $1$ (or $\top$, if interpreted as a Boolean) at any given index implies that the validator assures they are contributing to its availability.\footnote{This is a ``soft'' implication since there is no consequence on-chain if dishonestly reported. For more information on this implication see section \ref{sec:assurance}.} Formally:
\begin{align}
  \mathbf{E}_A \in \lseq\ltuple\isa{a}{\H}\ts\isa{f}{\mathbb{B}_\mathsf{C}}\ts\isa{v}{\N_\mathsf{V}}\ts\isa{s}{\mathbb{E}}\rtuple\rseq_\mathsf{:V}
\end{align}

The assurances must all be anchored on the parent and ordered by validator index:
\begin{align}
  \forall a &\in \mathbf{E}_A : a_a = \mathbf{H}_p \\
  \forall i &\in \{ 1\dots|\mathbf{E}_A| \} : \mathbf{E}_A[i - 1]_v < \mathbf{E}_A[i]_v
\end{align}

The signature must be one whose public key is that of the validator assuring and whose message is the serialization of the parent hash $\mathbf{H}_p$ and the aforementioned bitstring:
\begin{align}
  &\forall a \in \mathbf{E}_A : a_s \in \sig{\kappa[a_v]_e}{\mathsf{X}_A\frown \mathcal{H}(\mathbf{H}_p, a_f)} \\
  &\mathsf{X}_A = \token{\$jam\_available}
\end{align}

A bit may only be set if the corresponding core has a report pending availability on it:
\begin{align}
  \forall a \in \mathbf{E}_A : \forall c \in \N_\mathsf{C} : a_f[c] \implies \rho^\dagger[c] \ne \none
\end{align}

\subsubsection{Available Reports}
A work-report is said to become \emph{available} if and only if there are a clear \nicefrac{2}{3} super-majority of validators who have marked its core as set within the block's assurance extrinsic. Formally, we define the series of available work-reports $\mathbf{R}$ as:
\begin{align}\label{eq:availableworkreports}
  \mathbf{R} &\equiv \left[\rho^\dagger[c]_w\,\middle\vert\,c \orderedin \N_\mathsf{C},\;\sum_{a \in \mathbf{E}_A}\!a_v[c]\,>\,\nicefrac{2}{3}\,\mathsf{V}\right]
\end{align}

This value is utilized in the definition of both $\delta'$ and $\rho^\ddagger$ which we will define presently as equivalent to $\rho^\dagger$ except for the removal of items which are now available:
\begin{align}
  \forall c \in \N_C: \rho^\ddagger[c] \equiv \begin{cases}
    \none &\when\rho[c]_w \in \mathbf{R}\\
    \rho^\dagger[c] &\otherwise
  \end{cases}
\end{align}











\subsection{Guarantor Assignments}\label{sec:coresandvalidators}

Every block, each core has some particular number of validators uniquely assigned to it assigned to guarantee work reports for it. With $\mathsf{V} = 1,023$ validators and $\mathsf{C} = 341$ cores, this results in exactly $\nicefrac{\mathsf{V}}{\mathsf{C}} = 3$ validators per core. The Ed25519 keys of these validators are denoted by $\mathbf{G}$:
\begin{equation}
  \mathbf{G} \in \seq{\seq{\H_E}_{\nicefrac{\mathsf{V}}{\mathsf{C}}}}_{\mathsf{C}}
\end{equation}

We determine the core to which any given validator is assigned through a shuffle using epochal entropy and a periodic rotation to help guard the security and liveness of the network. We use $\eta_2$ for the epochal entropy rather than $\eta_1$ to avoid the possibility of fork-magnification where uncertainty about chain state at the end of an epoch could give rise to two established forks before it naturally resolves.

We define the permute function $P$, the rotation function $R$ and finally the guarantor assignments  $\mathbf{G}$ as follows:
\begin{align}
  P(e, t) &\equiv R\left(\mathcal{F}\left(\left[\ffrac{\mathsf{V}\cdot i}{\mathsf{C}} \,\middle\mid\, i \orderedin \N_\mathsf{V}\right], e\right), \ffrac{t \bmod \mathsf{E}}{\mathsf{R}}\right)\!\!\!\!\!\! \\[3pt]
  R(\mathbf{c}, n) &\equiv [(x + n) \bmod \mathsf{C} \mid x \orderedin \mathbf{c}]\\
  \forall c \in \N_C : \mathbf{G} &\equiv [ \kappa'_i \mid i \orderedin \N_V, P(\eta_2', \tau')_i = c ]
\end{align}

We also define $\mathbf{G}^*$, which is equivalent to the value $\mathbf{G}$ as it would have been under the previous rotation:
\begin{align}
  \label{eq:priorassignments}
  \forall c \in \N_C : \mathbf{G}^* &\equiv [ \mathbf{k}_i \mid i \orderedin \N_V, P(e, \tau' - \mathsf{R})_i = c ]\\
  \where e &= \begin{cases}
    (\eta'_2, \kappa) &\when \displaystyle\ffrac{\tau' - \mathsf{R}}{\mathsf{E}} = \ffrac{\tau'}{\mathsf{E}}\\
    (\eta'_3, \lambda) &\otherwise
  \end{cases}
\end{align}













\subsection{Work Report Guarantees}\label{sec:workreportguarantees}

We begin by defining the guarantees extrinsic, $\mathbf{E}_G$, a series of \emph{guarantees}, at most one for each core, each of which is a tuple of a core index, \emph{work-report}, a credential $a$ and its corresponding timeslot $t$. Formally:
\begin{align}
  \mathbf{E}_G &\in \seq{\tuple{
    c \in \N_\mathsf{C},\, w \in \mathbb{W},\, t \in \N_T,\, a \in \seq{\mathbb{E}?}_3
  }}_{:\mathsf{C}} \\
  \forall g &\in \mathbf{E}_G : (g_a)_{\dots2} \in \seq{\mathbb{E}}_2
\end{align}

The credential is itself a sequence of three tuples of a signature and a validator index $\mathbb{E}$. Only the final item may legitimately be $\none$.

The core index of each guarantee must be in ascending order:
\begin{align}
  \mathbf{E}_G = \orderby{i_c}{i \in \mathbf{E}_G}
\end{align}

Credentials may only have one missing signature:
\begin{equation}
  \forall g \in \mathbf{E}_G : |\{x \in g_a : x \ne \none\}| \ge 2
\end{equation}

The signature must be one whose public key is that of the validator identified in the credential, and whose message is the serialization of the core index and the work-report. The signing validators must be assigned to the core in question in either this block $\mathbf{G}$ if the timeslot for the guarantee is in the same rotation as this block's timeslot, or in the most recent previous set of assignments, $\mathbf{G}^*$:
\begin{align}\label{eq:guarantorsig}
    \nonumber&\forall g \in \mathbf{E}_G:\\[-9pt]
    &\quad\forall i \in \N_3, g_a[i] \ne \none: \left\{\,\begin{aligned}
      &a_s \in \sig{\mathbf{k}[g_c]_i}{\mathsf{X}_G\frown\mathcal{H}(g_c, g_r)}\\
      &\where \mathbf{k} = \begin{cases}
        \mathbf{G} \!\!\!\!&\when \ffrac{\tau'}{\mathsf{R}} = \ffrac{g_t}{\mathsf{R}} \\
        \mathbf{G}^* \!\!\!\!&\otherwise
      \end{cases}
    \end{aligned}\right.\!\!\!\!\!\!
\end{align}
\begin{equation}
  \mathsf{X}_G = \token{\$jam\_guarantee}
\end{equation}

No reports may be placed on cores with a report pending availability on it unless it has timed out. In the latter case, $\mathsf{U} = 5$ slots must have elapsed after the report was made. A report is invalid if the authorizer hash is not present in the authorizer pool of the core on which the work is reported. Formally:
\begin{equation}\label{eq:reportcoresareunusedortimedout}
  \forall g \in \mathbf{E}_G :\left\{\;\begin{aligned}
    &\rho^\ddagger[g_c] = \none \vee \mathbf{H}_t \ge \rho^\ddagger[g_c]_t + \mathsf{U}\ , \\
    &g_a \in \alpha[g_c]
  \end{aligned}\right.
\end{equation}

We denote $\mathbf{w}$ to be the set of work-reports in the present extrinsic $\mathbf{E}$:
\begin{align}
  \text{let}\ \mathbf{w} = \{ g_w \mid g \in \mathbf{E}_G \}
\end{align}

We specify the maximum total accumulation gas requirement a work-report may imply as $\mathsf{G}_A$, and we require the sum of all services' minimum gas requirements to be no greater than this:
\begin{align}
  \forall w \in \mathbf{w}: \sum_{s \in (w_r)_s}{\delta[s]_m}\ \le\ \mathsf{G}_A
\end{align}












\subsubsection{Contextual Validity of Reports}\label{sec:contextualvalidity}

For convenience, we define two equivalences $\mathbf{x}$ and $\mathbf{p}$ to be, respectively, the set of all contexts and work-package hashes within the extrinsic:
\begin{equation}
    \using \mathbf{x} \equiv \{ w_x \mid w \in \mathbf{w} \}\ ,\quad
    \mathbf{p} \equiv \{ (w_s)_h \mid w \in \mathbf{w} \}
\end{equation}

There must be no duplicate work-package hashes (\ie two work-reports of the same package). Therefore, we require the cardinality of $\mathbf{p}$ to be the length of the work-report sequence $\mathbf{w}$:
\begin{equation}
  |\mathbf{p}| = |\mathbf{w}|
\end{equation}

We require that the anchor block be within the last $\mathsf{H}$ blocks and that its details be correct by ensuring that it appears within our most recent blocks $\beta$:
\begin{align}
  \forall x \in \mathbf{x} : \exists y \in \beta : x_a = y_h \wedge x_s = y_s \wedge x_b = \mathcal{H}_K(\se_M(y_\mathbf{b}))\!\!\!\!\!\!
\end{align}

We require that each lookup-anchor block be within the last $\mathsf{L}$ timeslots:
\begin{align}
  \forall x \in \mathbf{x} :\ x_t \ge \mathbf{H}_t - \mathsf{L}
\end{align}

We also require that we have a record of it; this is one of the few conditions which cannot be checked purely with on-chain state and must be checked by virtue of retaining the series of the last $\mathsf{L}$ headers as the ancestor set $\mathbf{A}$. Since it is determined through the header chain, it is still deterministic and calculable. Formally:
\begin{align}
  \forall x \in \mathbf{x} :\ \exists h \in \mathbf{A}: h_t = x_t \wedge \mathcal{H}(h) = x_h
\end{align}

We require that the work-package of the report not be the work-package of some other report made in the past. Since the work-package implies the anchor block, and the anchor block is limited to the most recent blocks, we need only ensure that the work-package not appear in our recent history:
\begin{equation}
  \forall p \in \mathbf{p}, \forall x \in \beta : p \not\in x_\mathbf{p}
\end{equation}

We require that the prerequisite work-package, if present, be either in the extrinsic or in our recent history:
\begin{equation}\begin{aligned}
  &\forall w \in \mathbf{w}, (w_x)_p \ne \none :\\
  &\quad(w_x)_p \in \mathbf{p} \cup \{ x \mid x \in b_\mathbf{p} ,\, b \in \beta \}
\end{aligned}\end{equation}

We require that all work results within the extrinsic predicted the correct code hash for their corresponding service:
\begin{align}\label{eq:reportcodesarecorrect}
  \forall w \in \mathbf{w}, \forall r \in w_r : r_c = \delta[r_s]_c
\end{align}









\subsection{Transitioning for Reports}

We define $\rho'$ as being equivalent to $\rho^\ddagger$, except where the extrinsic replaced an entry. In the case an entry is replaced, the new value includes the present time $\tau'$ allowing for the value may be replaced without respect to its availability once sufficient time has elapsed (see equation \ref{eq:reportcoresareunusedortimedout}).
\begin{equation}
\begin{aligned}
    \forall c \in \N_\mathsf{C} : \rho'[c] &\equiv \begin{cases}
      \tup{w\ts\is{g}{G(a)}\ts\is{t}{\tau'}}\!\!\!\!\! &\when \exists \tup{c, w, a} \in \mathbf{E}_G \\
      \rho^\ddagger[c] &\otherwise
    \end{cases}\!\!\!\!\!\!\!\!\! \\
    \where G(a) &\equiv \{\kappa[v]_e \mid (s, v) \in a \}
\end{aligned}
\end{equation}

This concludes the section on reporting and assurance. We now have a complete definition of $\rho'$ together with $\mathbf{R}$ to be utilized in section \ref{sec:accumulation}, describing the portion of the state transition happening once a work-report is guaranteed and made available.
